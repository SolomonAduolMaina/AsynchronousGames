\documentclass{article}

\usepackage{amsmath, amssymb, amsthm, verbatim, enumerate, 
graphicx, centernot, tikz, array, tikz-cd, extarrows, cleveref,
mathrsfs, mathtools, bussproofs, stmaryrd, enumitem, listings}

\begin{document}
\section{The Language Non-Determnistic TSO-Algol}
\subsection{Syntax}
The types of ND-TSO-Algol terms are generated by the grammar,
$$\tau := com \; | \; nat \; | \; var \; | \; \tau \rightarrow \tau' \; | \;
\tau \times \tau'$$ 
while the language ND-TSO-Algol is generated by the grammar,
\begin{align*}
e \; := \; &x \; | \; n \; | \; \; succ \; e \; | \; e \; e' \; | \;
\lambda (x:\tau).e \; | \; fix \; e \; | \; if0 \; e \; e' \; e'' \; | \; (e,
e') \; | \; new \; e \; \\
&assign \; e \; e' \; | \; deref \; e \; |
\; fork \; e \; e' \; | \; skip \; | \; e \; e' \; | \; mkvar \; e \; e' \; | \;
e \; or \; e'
\end{align*}
where $x,v$ range of strings, $\tau$ rangers over types, and $n$ ranges over
natural numbers.
\subsection{Typing Rules}
A typing rule takes the form 
$$(x_1 : \tau_1, \ldots, x_n:\tau_n \vdash e : \tau), k$$
where the $x_i$'s are string identifiers, the $\tau_i$'s and $\tau$ are types,
$k$ is a natural number, and $e$ is a term. Here $k$ denotes the identifier of
the thread in which $e$ is running. The $k$ is used in the following typing
rule:
\begin{prooftree}
\AxiomC{$(\Gamma \vdash e: nat), k+1$}
\AxiomC{$(\Gamma \vdash e': nat), k$}
\BinaryInfC{$(\Gamma \vdash fork \; e \; e': nat), k$}
\end{prooftree}
The other typing rules are as expected and use the same thread identifier in the
premises and conclusion.
\subsection{Operational Semantics}
Small step rules take the familiar form
$$(\Gamma, e) \mapsto (\Gamma', e')$$
where $\Gamma, \Gamma'$ are typing contexts and $e, e'$ are terms. The new
evaluation rules for parallel composition are
\begin{prooftree}
\AxiomC{$(\Gamma, e) \mapsto (\Gamma', e')$}
\UnaryInfC{$(\Gamma, fork \; e \; e'') \mapsto (\Gamma', fork \; e' \; e'')$}
\end{prooftree}
\begin{prooftree}
\AxiomC{$(\Gamma, e) \mapsto (\Gamma', e')$}
\UnaryInfC{$(\Gamma, fork \; e'' \; e) \mapsto (\Gamma', fork \; e'' \; e')$}
\end{prooftree}
\begin{prooftree}
\AxiomC{$v$ is a value}
\UnaryInfC{$(\Gamma, fork \; v \; e) \mapsto (\Gamma, e)$}
\end{prooftree}
\section{Game Semantics of TSO-Algol}
A well-typed term $(x_1 : \tau_,\ldots, x_n : \tau_n \vdash e : \tau), k$ is
modelled by a strategy on the game $(\llbracket \tau_1 \rrbracket \times
\ldots \times \llbracket \tau_n \rrbracket) \Rightarrow \tau$.

Recall that previously, the type $var$ was the product $\mathbb{N} \times
\mathbb{N}^{\bot}$, where $\mathbb{N}$ is the arena with an Opponent move $read$
and a Player move $n$ for every $n \in \mathbb{N}$, while $\mathbb{N}^{\bot}$ is
the arena with a Player move $write_n$ for every $n \in \mathbb{N}$ and an
Opponent move $ok$. To model TSO behaviour we need to have a more complicated
type for var. Let $\mathbf{Var} = \mathbb{N} \times
\mathbb{N}^{\bot}$, and define the new type of $var$ by
$$var = (\mathbb{N} \longrightarrow (\mathbf{Var} \times \mathbf{Var})) \times
\mathbf{Var}$$
The first factor $(\mathbb{N} \longrightarrow (\mathbf{Var} \times
\mathbf{Var}))$ denotes the local versions of the variable while the second factor
$\mathbf{Var}$ denotes the global version of the variable.

Let $\mathbf{assign}$ and $\mathbf{deref}$ stand for the old assignment and
dereference statements. We define flushing events as follows:
\begin{align*}
choose\_n : \; nat = \; &new \; \lambda (x:\mathbf{Var}). \;
fix(\lambda(f:nat\rightarrow nat).\; (!x) \; or \; (x:=x+1 \; ; \; f \; x))\\
flush \; (e:var) \; : \; com \; = \; &let \; k := choose\_n \; in \\
&if0 \; (\mathbf{deref} \;(fst \; ((fst \; e) \; k))) \; done \; \\
&(\mathbf{assign} \; (snd \; e) \; (snd \; ((fst \; e) \; k)) \; ; \;
\mathbf{assign} \; (fst \; ((fst \; e) \; k)) \; 0)\\
flush\_many \; (e:var) \; = \; &fix(\lambda(f:com\rightarrow com).\; done \; or
\; (flush \; e \; ; \; f \; e))
\end{align*}
Having done this, we can define the new $assign$ and $deref$ events as follows;
$\begin{array}{l l l}
\llbracket (\Gamma \vdash assign \; e \; e'), k \rrbracket &\equiv
& \llbracket \Gamma \vdash (\mathbf{assign} \; (snd \; ((fst \; e) \; k)) \;
e' \; ; \;\\
& & \mathbf{assign} \; (fst \; ((fst \; e) \; k)) \; 1) \; ; \; flush\_many \; e
\rrbracket\\
\llbracket (\Gamma \vdash deref \; e), k \rrbracket &\equiv
&\llbracket \Gamma \vdash (if0 \; (\mathbf{deref} \; (fst \; ((fst \; e) \; k)))
\; (\mathbf{deref} \; (snd \; e)) \\
& &(\mathbf{deref} \; (snd \; ((fst \; e) \; k)))) \; ; \; flush\_many \; e
\rrbracket
\end{array}$

As for $fork$, the strategy $\llbracket fork \rrbracket : (\llbracket \mathbb{N}
\rrbracket_0 \times \llbracket \mathbb{N} \rrbracket_1) \Rightarrow
\llbracket \mathbb{N} \rrbracket_2$ is the strategy whose complete plays are
given by
\begin{align*}
&q_2 \cdot (q_0 \cdot n_0 \cdot q_1 \cdot n_1) \cdot n_1\\
&q_2 \cdot (q_0 \cdot q_1 \cdot n_0 \cdot n_1) \cdot n_1\\
&q_2 \cdot (q_0 \cdot q_1 \cdot n_1 \cdot n_0) \cdot n_1\\
&q_2 \cdot (q_1 \cdot n_1 \cdot q_0 \cdot n_0) \cdot n_1\\
&q_2 \cdot (q_1 \cdot q_0 \cdot n_1 \cdot n_0) \cdot n_1\\
&q_2 \cdot (q_1 \cdot q_0 \cdot n_0 \cdot n_1) \cdot n_1
\end{align*}
Note that this strategy does not satisfy alternation; it is a \textit{saturated}
strategy as described in the paper ``Angelic Semantics of Fine-Grained
Concurrency''. This basically means that the strategy allows for interleaving of
independent threads of computation.
\end{document}