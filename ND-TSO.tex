\documentclass{article}

\usepackage{amsmath, amssymb, amsthm, verbatim, enumerate, 
graphicx, centernot, tikz, array, tikz-cd, extarrows, cleveref,
mathrsfs, mathtools, bussproofs, stmaryrd, enumitem, listings}

\begin{document}
\section{The Language Non-Deterministic TSO-Algol}
\subsection{Syntax}
The types of ND-TSO-Algol terms are generated by the grammar,
$$\tau := com \; | \; nat \; | \; var_{\tau} \; | \; \tau \rightarrow \tau' \; | \;
\tau \times \tau'$$
while the language ND-TSO-Algol is generated by the grammar,
\begin{align*}
e \; := \; &x \; | \; n \; | \; \; succ \; e \; | \; e \; e' \; | \;
\lambda (x:\tau).e \; | \; if0 \; e \; e' \; e'' \; | \; (e,
e') \; | \; new_{\tau} \; e \; | \; e;e'\\
&assign \; e \; e' \; | \; deref \; e \; |
\; fork \; e \; e' \; | \; skip \; | \; e \; e' \; | \; mkvar \; e \; e' \; | \;
e \; or \; e'
\end{align*}
where $x,v$ range of strings, $\tau$ rangers over types, and $n$ ranges over
natural numbers. Note that $new_{\tau}$ is indexed by types i.e. ND-TSO allows for general references, hence the $fix$ combinator is omitted.
\subsection{Typing Rules}
A typing rule takes the form 
$$(x_1 : \tau_1, \ldots, x_n:\tau_n \vdash e : \tau), k$$
where the $x_i$'s are string identifiers, the $\tau_i$'s and $\tau$ are types,
$k$ is a natural number, and $e$ is a term. Here $k$ denotes the identifier of
the thread in which $e$ is running. The $k$ is used in the following typing
rule:
\begin{prooftree}
\AxiomC{$(\Gamma \vdash e: nat), k+1$}
\AxiomC{$(\Gamma \vdash e': nat), k$}
\BinaryInfC{$(\Gamma \vdash fork \; e \; e': nat), k$}
\end{prooftree}
The other typing rules are as expected and use the same thread identifier in the
premises and conclusion.
\subsection{Operational Semantics}
Small step rules take the familiar form
$$(\Gamma, e) \mapsto (\Gamma', e')$$
where each of $\Gamma, \Gamma'$ is a partial map from string identifiers to values, and $e, e'$ are terms. The new
evaluation rules for parallel composition are
\begin{prooftree}
\AxiomC{$(\Gamma, e) \mapsto (\Gamma', e')$}
\UnaryInfC{$(\Gamma, fork \; e \; e'') \mapsto (\Gamma', fork \; e' \; e'')$}
\end{prooftree}
\begin{prooftree}
\AxiomC{$(\Gamma, e) \mapsto (\Gamma', e')$}
\UnaryInfC{$(\Gamma, fork \; e'' \; e) \mapsto (\Gamma', fork \; e'' \; e')$}
\end{prooftree}
\begin{prooftree}
\AxiomC{$v$ is a value}
\UnaryInfC{$(\Gamma, fork \; v \; e) \mapsto (\Gamma, e)$}
\end{prooftree}
\section{Game Semantics of TSO-Algol}
A well-typed term $(x_1 : \tau_,\ldots, x_n : \tau_n \vdash e : \tau), k$ is
modelled by a strategy on the game $(\llbracket \tau_1 \rrbracket \times
\ldots \times \llbracket \tau_n \rrbracket) \Rightarrow \tau$.

To model TSO behaviour we need to have a more complicated
type for references. Let $\mathbf{Var}_{\tau}$ denote the old type of $\tau$ reference, and define the new type of $var_{\tau}$ by
$$var_{\tau} = \mathbf{Var}_{(\mathbb{N} \longrightarrow (\mathbb{N} \times \tau))} \times
\mathbf{Var}_{\tau}$$
The first factor $\mathbf{Var}_{(\mathbb{N} \longrightarrow (\mathbb{N} \times \tau))}$ denotes the local versions of the variable while the second factor
$\mathbf{Var}_{\tau}$ denotes the global version of the variable.

We define flushing events as follows:
\begin{align*}
choose\_n : \; nat = \; &new \; \lambda (x:\mathbf{Var}_{\mathbb{N}}). \;
fix(\lambda f.\; (!x) \; or \; (x:=x+1 \; ; \; f \; x))\\
invalidate \; (v : \mathbf{Var}_{\mathbb{N} 
 \rightarrow (\mathbb{N} \times \tau)}) \; (k : \mathbb{N}) \; = \; &v :=
 \lambda \; (n:\mathbb{N}). if \; n=k \; then \; (0, snd \; ((!v) \; k)) \; else \; (!v) \; k\\
flush \; (e:var_{\tau}) \; : \; com \; = \; &let \; k := choose\_n \; in \\
&if \; ((fst \; (!(fst \; e) \; k))) = 0 \; then \; done \; else \\
&invalidate \; v \; k; (snd \; e):= (snd \; (!(fst \; e) \; k))\\
flush\_many \; (e:var_{\tau}) \; = \; &fix(\lambda(f:com\rightarrow com).\; done \; or
\; (flush \; e \; ; \; f \; e))\\
set\_local \; (v : \mathbf{Var}_{\mathbb{N} 
 \rightarrow (\mathbb{N} \times \tau)}) \; (k : \mathbb{N}) \; (e : \tau) \; = \; 
 & v:= \lambda \; (n:\mathbb{N}). if \; n=k \; then \; (1, e) \; else \; (!v) \; k\\
local\_read \; (v : \mathbf{Var}_{\mathbb{N} 
 \rightarrow (\mathbb{N} \times \tau)}) \; (k : \mathbb{N}) \; = \; &snd \; ((!v) \; k)
\end{align*}
Having done this, we can define the new $assign$ and $deref$ events as follows;
$\begin{array}{l l l}
\llbracket (\Gamma \vdash assign \; e \; e'), k \rrbracket &\equiv
& \llbracket \Gamma \vdash set\_local \; (fst \; e) \; k \; e' \; ; \; flush\_many \; e
\rrbracket\\
\llbracket (\Gamma \vdash deref \; e), k \rrbracket &\equiv
&\llbracket \Gamma \vdash 
(if \; ((fst \; (!(fst \; e) \; k))) = 0 \; then \; !(snd \; e) \; else\\
& &local\_read \; (fst \; e) \; k \;); flush\_many \; e
\rrbracket
\end{array}$

As for $fork$, the strategy $\llbracket fork \rrbracket : (\llbracket \mathbb{N}
\rrbracket_0 \times \llbracket \mathbb{N} \rrbracket_1) \Rightarrow
\llbracket \mathbb{N} \rrbracket_2$ is the strategy whose complete plays are
given by
\begin{align*}
&q_2 \cdot (q_0 \cdot n_0 \cdot q_1 \cdot n_1) \cdot n_1\\
&q_2 \cdot (q_0 \cdot q_1 \cdot n_0 \cdot n_1) \cdot n_1\\
&q_2 \cdot (q_0 \cdot q_1 \cdot n_1 \cdot n_0) \cdot n_1\\
&q_2 \cdot (q_1 \cdot n_1 \cdot q_0 \cdot n_0) \cdot n_1\\
&q_2 \cdot (q_1 \cdot q_0 \cdot n_1 \cdot n_0) \cdot n_1\\
&q_2 \cdot (q_1 \cdot q_0 \cdot n_0 \cdot n_1) \cdot n_1
\end{align*}
Note that this strategy does not satisfy alternation; it is a \textit{saturated}
strategy as described in the paper ``Angelic Semantics of Fine-Grained
Concurrency''. This basically means that the strategy allows for interleaving of
independent threads of computation.
\end{document}